\documentclass[11pt,a4paper]{article}

\usepackage[utf8]{inputenc}
\usepackage[T1]{fontenc}
\usepackage{lmodern}
\usepackage[margin=1in]{geometry}
\usepackage{graphicx}
\usepackage{booktabs}
\usepackage{amsmath,amssymb}
\usepackage{hyperref}
\usepackage{xcolor}
\usepackage{float}
\usepackage{caption,subcaption}
\usepackage{enumitem}
\usepackage{algorithm,algpseudocode}

\definecolor{success}{RGB}{40, 167, 69}
\definecolor{primary}{RGB}{30, 58, 95}

\hypersetup{colorlinks=true,linkcolor=primary,citecolor=primary,urlcolor=primary}

\title{\textbf{SI Signal Discovery: Cross-Market Validation of Specialization Index as a Trading Signal}\\[0.5em]\large Technical Report}
\author{Yuhao Li\\\textit{University of Pennsylvania}\\\texttt{li88@sas.upenn.edu}}
\date{January 17, 2026}

\begin{document}
\maketitle

\begin{abstract}
We investigate whether the Specialization Index (SI), a metric derived from emergent agent specialization in competitive trading environments, correlates with meaningful market features. Through pre-registered hypothesis testing across 11 assets spanning 4 market types (crypto, forex, stocks, commodities), we find that SI significantly correlates with 17 features at $|r| > 0.15$ after FDR correction. Correlations replicate on holdout validation (51\%) and test (44\%) sets. We compare three regime detection methods and find that simple rule-based regimes yield the most consistent SI correlations (5\% sign flip rate). These results suggest SI captures meaningful market dynamics warranting further development as a trading signal.

\vspace{0.5em}
\noindent\textbf{Keywords:} Specialization Index, emergent behavior, multi-agent systems, regime detection, quantitative trading
\end{abstract}

\tableofcontents
\newpage

\section{Introduction}

\subsection{Motivation}
Traditional quantitative trading relies on hand-crafted indicators requiring extensive domain expertise. Recent advances in multi-agent systems demonstrate that useful structure can \textit{emerge} from competition alone. The Specialization Index (SI) measures how specialized agents become when competing for rewards. Prior work showed SI emerges spontaneously in learner populations with Cohen's $d > 20$ separation from baselines.

\subsection{Research Question}
\textbf{Primary Question:} What does SI correlate with in financial markets, and can it serve as a trading signal?

This is a \textit{discovery-first} approach---we systematically test correlations with 46 market features across multiple assets and markets before making claims about predictive power.

\subsection{Contributions}
\begin{enumerate}
    \item Pre-registered discovery study preventing p-hacking
    \item Cross-market validation across 4 market types
    \item Rigorous methodology: FDR correction, block bootstrap, train/val/test splits
    \item Regime analysis comparing rule-based, HMM, and GMM methods
    \item Open science: all code and data publicly available
\end{enumerate}

\section{Related Work}

\subsection{Emergent Specialization}
Emergent specialization has been studied in evolutionary computation, multi-agent reinforcement learning, and biological systems. Competition for limited resources naturally leads to niche differentiation without explicit diversity mechanisms.

\subsection{Regime Detection in Finance}
Market regime detection approaches include:
\begin{itemize}[noitemsep]
    \item \textbf{Hidden Markov Models (HMM)}: Probabilistic regime inference (Hamilton 1989)
    \item \textbf{Gaussian Mixture Models (GMM)}: Data-driven clustering
    \item \textbf{Rule-based}: Simple thresholds on volatility or trend strength
\end{itemize}

\section{Methodology}

\subsection{Specialization Index Definition}
\begin{equation}
\text{SI} = 1 - \frac{1}{N} \sum_{i=1}^{N} H_{\text{norm}}(\mathbf{a}_i)
\end{equation}
where $N$ is the number of agents, $\mathbf{a}_i$ is agent $i$'s niche affinity vector, and $H_{\text{norm}}$ is normalized entropy. High SI ($\rightarrow 1$) indicates specialists; low SI ($\rightarrow 0$) indicates generalists.

\subsection{Pre-Registration}
All hypotheses were pre-registered before analysis:
\begin{itemize}[noitemsep]
    \item Report all results including null findings
    \item No post-hoc hypothesis changes
    \item Use Benjamini-Hochberg FDR correction at $\alpha = 0.05$
    \item Validate on holdout sets before claiming significance
\end{itemize}

\subsection{Data}
\begin{table}[H]
\centering
\caption{Data Coverage}
\begin{tabular}{llccc}
\toprule
\textbf{Market} & \textbf{Assets} & \textbf{Frequency} & \textbf{Period} & \textbf{N} \\
\midrule
Crypto & BTC, ETH, SOL & Daily & 5 years & $\sim$1,800 \\
Forex & EUR/USD, GBP/USD, USD/JPY & Daily & 5 years & $\sim$1,300 \\
Stocks & SPY, QQQ, AAPL & Daily & 5 years & $\sim$1,250 \\
Commodities & Gold, Oil & Daily & 5 years & $\sim$1,250 \\
\bottomrule
\end{tabular}
\end{table}

Data split temporally: train (70\%), validation (15\%), test (15\%) with 7-day purging gap.

\subsection{Statistical Analysis}
\begin{itemize}[noitemsep]
    \item Spearman rank correlation for monotonic relationships
    \item Benjamini-Hochberg FDR correction for 286 tests
    \item Block bootstrap (1,000 iterations) for confidence intervals
    \item Effect size threshold: $|r| > 0.10$ meaningful, $|r| > 0.15$ strong
\end{itemize}

\section{Results}

\subsection{Discovery: SI Correlates}
After FDR correction and effect size filtering, we identified \textbf{84 meaningful correlations} across 256 tests. \textbf{17 features} showed $|r| > 0.15$.

\begin{figure}[H]
\centering
\includegraphics[width=0.95\textwidth]{figures/fig2_top_correlates.png}
\caption{Top SI correlates ranked by number of markets where significant.}
\end{figure}

\textbf{Key findings:}
\begin{itemize}[noitemsep]
    \item \textbf{ADX}: $r = +0.15$ to $+0.23$ across all 4 markets
    \item \textbf{Bollinger Band Width}: $r = +0.22$ to $+0.29$, 4 markets
    \item \textbf{RSI}: $r = +0.20$ to $+0.30$, 4 markets
    \item \textbf{Volatility}: $r = -0.15$ to $-0.23$, 4 markets
\end{itemize}

\textbf{Interpretation:} SI is higher when markets have clear trends, moderate volatility, and well-defined technical signals---``market readability.''

\subsection{Cross-Market Validation}

\begin{figure}[H]
\centering
\includegraphics[width=0.95\textwidth]{figures/fig1_cross_market_validation.png}
\caption{Validation and test confirmation rates by asset.}
\end{figure}

\begin{table}[H]
\centering
\caption{Confirmation Rates by Market}
\begin{tabular}{lccc}
\toprule
\textbf{Market} & \textbf{Assets} & \textbf{VAL Rate} & \textbf{TEST Rate} \\
\midrule
Crypto & 3/3 & 37.9\% & 24.8\% \\
Forex & 3/3 & 55.8\% & 52.5\% \\
Stocks & 3/3 & 54.0\% & 50.0\% \\
Commodities & 2/2 & 59.5\% & 52.4\% \\
\midrule
\textbf{Overall} & \textbf{11/11} & \textbf{51.1\%} & \textbf{44.2\%} \\
\bottomrule
\end{tabular}
\end{table}

\subsection{Regime Analysis}

\begin{figure}[H]
\centering
\includegraphics[width=0.95\textwidth]{figures/fig3_regime_comparison.png}
\caption{(A) Sign flip rate by method. (B) Agreement between methods.}
\end{figure}

\begin{table}[H]
\centering
\caption{Regime Detection Method Comparison}
\begin{tabular}{lccc}
\toprule
\textbf{Method} & \textbf{Flip Rate} & \textbf{Avg Duration} & \textbf{Recommendation} \\
\midrule
Rule-based & \textcolor{success}{\textbf{5.0\%}} & 3.5 days & Best for SI \\
GMM & 10.4\% & 4.5 days & Good alternative \\
HMM & 17.5\% & 27+ days & Too smooth \\
\bottomrule
\end{tabular}
\end{table}

\subsection{Exit Criteria}

\begin{figure}[H]
\centering
\includegraphics[width=0.8\textwidth]{figures/fig5_exit_criteria.png}
\caption{All exit criteria satisfied.}
\end{figure}

\begin{table}[H]
\centering
\caption{Exit Criteria Status}
\begin{tabular}{lccl}
\toprule
\textbf{Criterion} & \textbf{Required} & \textbf{Actual} & \textbf{Status} \\
\midrule
Features $|r| > 0.15$, FDR $< 0.05$ & $\geq 3$ & 17 & \textcolor{success}{PASS} \\
VAL and TEST $> 30\%$ & Both & 51\%/44\% & \textcolor{success}{PASS} \\
Replicate in $\geq 3$ assets & $\geq 3$ & 11 & \textcolor{success}{PASS} \\
Replicate in $\geq 2$ markets & $\geq 2$ & 4 & \textcolor{success}{PASS} \\
\bottomrule
\end{tabular}
\end{table}

\section{Discussion}

\subsection{Interpretation}
SI captures ``market readability''---conditions favoring specialized strategies:
\begin{enumerate}
    \item Clear trends (positive correlation with ADX)
    \item Moderate volatility (negative correlation with extreme vol)
    \item Well-defined technical signals (RSI, BB width)
\end{enumerate}

\subsection{Why Rule-Based Beats HMM}
\begin{enumerate}
    \item Alignment with strategy logic (trend strength relates to momentum vs mean-reversion)
    \item Responsiveness (3.5-day vs 27+ day regimes)
    \item HMM averages over heterogeneous conditions
\end{enumerate}

\subsection{Cross-Market Generalization}
SI correlations replicate across markets with different trading hours, volatility profiles, and participant composition---suggesting emergent specialization captures fundamental market dynamics.

\section{Limitations}
\begin{enumerate}
    \item Simple strategies only (momentum, mean-reversion, breakout)
    \item No market impact or slippage modeling
    \item Daily data only---intraday patterns not captured
    \item 21\% of correlations flip sign across regimes
\end{enumerate}

\section{Conclusion}

\textcolor{success}{\textbf{PRIMARY HYPOTHESIS SUPPORTED}}

\begin{enumerate}
    \item 17 features correlate with SI ($|r| > 0.15$, FDR $< 0.05$)
    \item Replicates on holdout data (VAL 51\%, TEST 44\%)
    \item Generalizes across 4 market types
    \item SI captures market readability
    \item Rule-based regimes work best
\end{enumerate}

\textbf{Future work:} Regime-specific SI strategies, real-time computation, regime prediction, portfolio allocation.

\section*{Acknowledgments}
The author thanks the University of Pennsylvania for computational resources.

\section*{Data Availability}
Code and data: \url{https://github.com/HowardLiYH/Emergent-Applications/tree/main/apps/trading}

\begin{thebibliography}{9}
\bibitem{hamilton1989} Hamilton, J.D. (1989). A new approach to the economic analysis of nonstationary time series. \textit{Econometrica}, 57(2):357--384.
\bibitem{cohen1988} Cohen, J. (1988). \textit{Statistical Power Analysis for the Behavioral Sciences}. Routledge.
\bibitem{ang2002} Ang, A. and Bekaert, G. (2002). Regime switches in interest rates. \textit{JBES}, 20(2):163--182.
\end{thebibliography}

\end{document}
